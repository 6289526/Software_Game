\documentclass{jarticle}
\usepackage[dvipdfmx]{graphicx}
\usepackage{here}

\begin{document}



\clearpage
\subsection{操作方法}
ゲームの操作は以下の入力機器のいずれかを用いる。
\begin{itemize}
    \item キーボード
    \item wiiリモコン
    \item バランスボード\&wiiリモコン
\end{itemize}

操作の種類は以下の通りである。
\begin{table}[H]
    \caption{操作方法1}
    \label{table:control1}
    \begin{center}
    \begin{tabular}{|c|c|c|}\hline
    操作 & キーボード & wiiリモコン\\ \hline
    加速 & W & 2 \\ \hline
    左右旋回 & AD & \verb+<->+ \\ \hline
    ジャンプ & Space & 振る or 1 \\ \hline
    ブロックの設置 & Enter & A \\ \hline
    ゲーム終了 & Escape& + \\\hline
    
    \end{tabular}
    \end{center}
\end{table}

\begin{table}[H]
    \caption{操作方法2}
    \label{table:control2}
    \begin{center}
    \begin{tabular}{|c|c|c|}\hline
    操作 & バランスボード\&wiiリモコン\\ \hline
    加速 & 2 \\ \hline
    左右旋回 & 体重移動 \\ \hline
    ジャンプ & 屈伸 or 振る or 1 \\ \hline
    ブロックの設置 & A \\ \hline
    ゲーム終了 & + \\\hline
    \end{tabular}
    \end{center}
\end{table}



\subsection{モジュール}
\subsubsection{サーバー}

以下に各モジュールの外部関数を説明する
\begin{enumerate}
    \item ネットワークモジュール
    \begin{table}[H]
        \label{table:fanc_s1-1}
        \begin{center}
            \begin{tabular}{|c||p{30em}|}\hline
                関数名&void SetupServer(int numCl, u\_short port)\\\hline
                機能&サーバーの初期設定を行う\\
                引数&numCl ―― クライアント数\\
                &port ―― ポート番号\\
                返り値&無し\\\hline
            \end{tabular}
        \end{center}
    \end{table}
    \begin{table}[H]
        \label{table:fanc_s1-2}
        \begin{center}
            \begin{tabular}{|c||p{30em}|}\hline
                関数名&int ControlRequests(void) \\\hline
                機能&クライアントからのリクエストに対応する\\
                引数&無し\\
                返り値&1:通信継続/0:通信終了\\\hline
            \end{tabular}
        \end{center}
    \end{table}
    \begin{table}[H]
        \label{table:fanc_s1-3}
        \begin{center}
            \begin{tabular}{|c||p{30em}|}\hline
                関数名&void RunCommand(int id, char com) \\\hline
                機能&コマンドの実行\\
                引数&id ―― 送信先のクライアントID\\
                &com ―― 送信するコマンド\\
                返り値&無し\\\hline
            \end{tabular}
        \end{center}
    \end{table}
    \begin{table}[H]
        \label{table:fanc_s1-4}
        \begin{center}
            \begin{tabular}{|c||p{30em}|}\hline
                関数名&void TerminateServer(void)  \\\hline
                機能&サーバーの終了処理を行う\\
                引数&無し\\
                返り値&無し\\\hline
            \end{tabular}
        \end{center}
    \end{table}
    \item マップモジュール
    \begin{table}[H]
        \label{table:fanc_s2-1}
        \begin{center}
            \begin{tabular}{|c||p{30em}|}\hline
                関数名&void LoadMapData(char* fileName)\\\hline
                機能&マップデータの読み込みを行う\\
                引数&fileName ―― マップデータファイル名\\
                返り値&無し\\\hline
            \end{tabular}
        \end{center}
    \end{table}
\end{enumerate}

\subsubsection{クライアント}

以下に各モジュールの外部関数を説明する
\begin{enumerate}
    \item ネットワークモジュール
    \begin{table}[H]
        \label{table:fanc_c1-1}
        \begin{center}
            \begin{tabular}{|c||p{30em}|}\hline
                関数名&void SetupClient(char *serverName, u\_short port) \\\hline
                機能&クライアントの初期設定を行う\\
                引数&serverName―― サーバー名\\
                &port ―― ポート番号\\
                返り値&無し\\\hline
            \end{tabular}
        \end{center}
    \end{table}
    \begin{table}[H]
        \label{table:fanc_c1-2}
        \begin{center}
            \begin{tabular}{|c||p{30em}|}\hline
                関数名&int ControlRequests(void) \\\hline
                機能&クライアントの制御を行う\\
                引数&無し\\
                返り値&1:通信継続/0:通信終了\\\hline
            \end{tabular}
        \end{center}
    \end{table}
    \begin{table}[H]
        \label{table:fanc_c1-3}
        \begin{center}
            \begin{tabular}{|c||p{30em}|}\hline
                関数名&int InCommand(char com) \\\hline
                機能&コマンドに対する処理を行う\\
                引数&com ―― コマンド\\
                返り値&1:通信継続/0:通信終了\\\hline
            \end{tabular}
        \end{center}
    \end{table}
    \begin{table}[H]
        \label{table:fanc_c1-4}
        \begin{center}
            \begin{tabular}{|c||p{30em}|}\hline
                関数名&void TerminateClient(void)  \\\hline
                機能&クライアントの終了処理を行う\\
                引数&無し\\
                返り値&無し\\\hline
            \end{tabular}
        \end{center}
    \end{table}
    
    \item システムモジュール
    \begin{table}[H]
        \label{table:fanc_s2-1}
        \begin{center}
            \begin{tabular}{|c||p{30em}|}\hline
                関数名&const PlayerData* GetPlayerData()\\\hline
                機能&プレイヤーデータ配列の先頭ポインタを返す\\
                引数&無し\\
                返り値&プレイヤーデータ配列の先頭ポインタ\\\hline
            \end{tabular}
        \end{center}
    \end{table}

    \item マップモジュール
    \begin{table}[H]
        \label{table:fanc_s3-1}
        \begin{center}
            \begin{tabular}{|c||p{30em}|}\hline
                関数名&SetMapData(int mapW,int mapH, int mapD, \\
                &int terrainData[MAP\_SIZE\_W][MAP\_SIZE\_H][MAP\_SIZE\_D])\\\hline
                機能&マップデータの初期設定を行う\\
                引数&mapW ―― マップの幅\\
                &mapH ―― マップの高さ\\
                &mapD ―― マップの奥行\\
                &terrainData ―― マップの地形データ\\
                返り値&無し\\\hline
            \end{tabular}
        \end{center}
    \end{table}

    \item 入力モジュール
    \begin{table}[H]
        \label{table:fanc_c4-1}
        \begin{center}
            \begin{tabular}{|c||p{30em}|}\hline
                関数名&virtual void GetInput(SDL\_Event event)\\\hline
                機能&ユーザー入力を受け取る\\
                引数&event ―― SDLのイベント\\
                返り値&無し\\\hline
            \end{tabular}
        \end{center}
    \end{table}
    \begin{table}[H]
        \label{table:fanc_c4-2}
        \begin{center}
            \begin{tabular}{|c||p{30em}|}\hline
                関数名&InputType GetInputType()\\\hline
                機能&現在の入力情報を返す\\
                引数&無し\\
                返り値&\_Input: 現在の入力情報\\\hline
            \end{tabular}
        \end{center}
    \end{table}
    \item グラフィックモジュール
    \begin{table}[H]
        \label{table:fanc_c5-1}
        \begin{center}
            \begin{tabular}{|c||p{30em}|}\hline
                関数名&void InitGraphic()\\\hline
                機能&グラフィックモジュールの初期設定を行う\\
                引数&無し\\
                返り値&無し\\\hline
            \end{tabular}
        \end{center}
    \end{table}
    \begin{table}[H]
        \label{table:fanc_c5-1}
        \begin{center}
            \begin{tabular}{|c||p{30em}|}\hline
                関数名&void Disp()\\\hline
                機能&グラフィックの表示を行う\\
                引数&無し\\
                返り値&無し\\\hline
            \end{tabular}
        \end{center}
    \end{table}
    
\end{enumerate}

\subsection{コマンドプロトコル}

\begin{enumerate}
    \item サーバー→クライアントのプロトコル
    \begin{table}[H]
        \label{table:command1-1}
        \begin{center}
            \begin{tabular}{|c||p{30em}|}\hline
                コマンド&M(PlayerData[])\\\hline
                機能&移動処理\\
                引数&全プレイヤーデータ\\
                対象&全クライアント\\
                タイミング&毎フレーム\\\hline
            \end{tabular}
        \end{center}
    \end{table}
    \begin{table}[H]
        \label{table:command1-2}
        \begin{center}
            \begin{tabular}{|c||p{30em}|}\hline
                コマンド&P(PlaceData)\\\hline
                機能&ブロックの設置\\
                引数&設置されたブロックのデータ\\
                対象&全クライアント\\
                タイミング&設置時\\\hline
            \end{tabular}
        \end{center}
    \end{table}
    \begin{table}[H]
        \label{table:command1-3}
        \begin{center}
            \begin{tabular}{|c||p{30em}|}\hline
                コマンド&D\\\hline
                機能&設置不可能\\
                引数&無し\\
                対象&クライアント\\
                タイミング&設置時\\\hline
            \end{tabular}
        \end{center}
    \end{table}
    \begin{table}[H]
        \label{table:command1-2}
        \begin{center}
            \begin{tabular}{|c||p{30em}|}\hline
                コマンド&G\\\hline
                機能&ゴールしたことを通知\\
                引数&なし\\
                対象&クライアント\\
                タイミング&ゴール時\\\hline
            \end{tabular}
        \end{center}
    \end{table}
    \begin{table}[H]
        \label{table:command1-2}
        \begin{center}
            \begin{tabular}{|c||p{30em}|}\hline
                コマンド&E(char[])\\\hline
                機能&エラーを通知\\
                引数&エラー内容\\
                対象&クライアント\\
                タイミング&エラー発生時\\\hline
            \end{tabular}
        \end{center}
    \end{table}
    \begin{table}[H]
        \label{table:command1-2}
        \begin{center}
            \begin{tabular}{|c||p{30em}|}\hline
                コマンド&F(PlayerData[])\\\hline
                機能&ゲームの終了を通知\\
                引数&全プレイヤーのデータ\\
                対象&全クライアント\\
                タイミング&全プレイヤーゴール時\\\hline
            \end{tabular}
        \end{center}
    \end{table}
    \begin{table}[H]
        \label{table:command1-3}
        \begin{center}
            \begin{tabular}{|c||p{30em}|}\hline
                コマンド&Q\\\hline
                機能&プログラムの強制終了\\
                引数&無し\\
                対象&全クライアント\\
                タイミング&任意\\\hline
            \end{tabular}
        \end{center}
    \end{table}
    
    \item クライアント→サーバーのプロトコル
    \begin{table}[H]
        \label{table:command2-1}
        \begin{center}
            \begin{tabular}{|c||p{30em}|}\hline
                コマンド&M(FloatPosition)\\\hline
                機能&移動処理\\
                引数&自身の移動後の座標\\
                対象&サーバー\\
                タイミング&毎フレーム\\\hline
            \end{tabular}
        \end{center}
    \end{table}
    \begin{table}[H]
        \label{table:command2-2}
        \begin{center}
            \begin{tabular}{|c||p{30em}|}\hline
                コマンド&P(PlaceData)\\\hline
                機能&ブロックの設置\\
                引数&設置するブロックのデータ\\
                対象&サーバー\\
                タイミング&設置時\\\hline
            \end{tabular}
        \end{center}
    \end{table}
    \begin{table}[H]
        \label{table:command2-3}
        \begin{center}
            \begin{tabular}{|c||p{30em}|}\hline
                コマンド&Q\\\hline
                機能&プログラムの強制終了\\
                引数&無し\\
                対象&サーバー\\
                タイミング&任意\\\hline
            \end{tabular}
        \end{center}
    \end{table}
\end{enumerate}


\end{document}
