\documentclass{jarticle}
\usepackage[dvipdfmx]{graphicx}
\usepackage{here}

\begin{document}



\clearpage
\subsection{操作方法}
ゲームの操作は以下の入力機器のいずれかを用いる。
\begin{itemize}
    \item キーボード
    \item wiiリモコン
    \item バランスボード\&wiiリモコン
\end{itemize}

操作の種類は以下の通りである。
\begin{table}[H]
    \caption{操作方法1}
    \label{table:control1}
    \begin{center}
    \begin{tabular}{|l|l|l|}\hline
    操作 & キーボード & wiiリモコン\\ \hline
    加速 & W & 2 \\ \hline
    左右旋回 & AD & \verb+<->+ \\ \hline
    ジャンプ & Space & 振る or 1 \\ \hline
    ブロックの設置 & Enter & A \\ \hline
    ゲーム終了 & Escape& + \\\hline
    
    \end{tabular}
    \end{center}
\end{table}

\begin{table}[H]
    \caption{操作方法2}
    \label{table:control2}
    \begin{center}
    \begin{tabular}{|l|l|l|}\hline
    操作 & バランスボード\&wiiリモコン\\ \hline
    加速 & 2 \\ \hline
    左右旋回 & 体重移動 \\ \hline
    ジャンプ & 屈伸 or 振る or 1 \\ \hline
    ブロックの設置 & A \\ \hline
    ゲーム終了 & + \\\hline
    \end{tabular}
    \end{center}
\end{table}

\subsection{ゲームデータの構造}
以下にゲームデータの構造を示す
\begin{table}[H]
    \caption{クライアントのデータ構造}
    \label{table:data1}
    \begin{center}
    \begin{tabular}{|l||l|l|}\hline
    データ型 & 変数名 & 内容 \\ \hline
    char & \verb+name[MAX_LEN_NAME]+ & クライアントの名前 \\ \hline
    FloatCube & pos & マップ上の場所 \\ \hline
    int & rank & 順位 \\ \hline
    bool & goal & ゴールしているか \\ \hline
    \end{tabular}
    \end{center}
\end{table}
\begin{table}[H]
    \caption{ネットワークモジュール用のクライアントの情報}
    \label{table:data2}
    \begin{center}
    \begin{tabular}{|l||l|l|}\hline
    データ型 & 変数名 & 内容 \\ \hline
    int & connect & クライアントがサーバーに接続しているか \\ \hline
    int & sock & 使用するソケット \\ \hline
    struct sockaddr\_in & addr & ソケットの設定 \\ \hline
    \end{tabular}
    \end{center}
\end{table}
\begin{table}[H]
    \caption{クライアントの座標}
    \label{table:data3}
    \begin{center}
    \begin{tabular}{|l||l|l|}\hline
    データ型 & 変数名 & 内容 \\ \hline
    float & x & x座標 \\ \hline
    float & y & y座標 \\ \hline
    float & z & z座標 \\ \hline
    \end{tabular}
    \end{center}
\end{table}
\begin{table}[H]
    \caption{直方体形を定義する構造体}
    \label{table:data4}
    \begin{center}
    \begin{tabular}{|l||l|l|}\hline
    データ型 & 変数名 & 内容 \\ \hline
    float & x & x座標 \\ \hline
    float & y & y座標 \\ \hline
    float & z & z座標 \\ \hline
    float & w & x方向の長さ \\ \hline
    float & h & y方向の長さ \\ \hline
    float & d & z方向の長さ \\ \hline
    \end{tabular}
    \end{center}
\end{table}
\begin{table}[H]
    \caption{マップ用のデータ}
    \label{table:data5}
    \centering
    \begin{tabular}{|l||l|l|}\hline
    データ型 & 変数名 & 内容 \\ \hline
    int & \verb+_TerrainData+  & マップデータ \\
    & \verb+[MAP_SIZE_W][MAP_SIZE_H][MAP_SIZE_D]+ & \\ \hline
    \verb+vector<PlaceData>+ & \_ObjectDatas & オブジェクトデータ \\ \hline
    int & \_MapW & マップの横幅 \\ \hline
    int & \_MapH & マップの横縦 \\ \hline
    int & \_MapD & マップのサイズ \\ \hline
    \end{tabular}
\end{table}

\subsection{モジュール}
\subsubsection{サーバー}

以下に各モジュールの外部関数を説明する
\begin{enumerate}
    \item ネットワークモジュール
    \begin{table}[H]
        \label{table:fanc_s1-1}
        \begin{center}
            \begin{tabular}{|c||p{30em}|}\hline
                関数名&void SetupServer(int numCl, u\_short port)\\\hline
                機能&サーバーの初期設定を行う\\
                引数&numCl ―― クライアント数\\
                &port ―― ポート番号\\
                返り値&無し\\\hline
            \end{tabular}
        \end{center}
    \end{table}
    \begin{table}[H]
        \label{table:fanc_s1-2}
        \begin{center}
            \begin{tabular}{|c||p{30em}|}\hline
                関数名&int ControlRequests(void) \\\hline
                機能&クライアントからのリクエストに対応する\\
                引数&無し\\
                返り値&1:通信継続/0:通信終了\\\hline
            \end{tabular}
        \end{center}
    \end{table}
    \begin{table}[H]
        \label{table:fanc_s1-3}
        \begin{center}
            \begin{tabular}{|c||p{30em}|}\hline
                関数名&void RunCommand(int id, char com) \\\hline
                機能&コマンドの実行\\
                引数&id ―― 送信先のクライアントID\\
                &com ―― 送信するコマンド\\
                返り値&無し\\\hline
            \end{tabular}
        \end{center}
    \end{table}
    \begin{table}[H]
        \label{table:fanc_s1-5}
        \begin{center}
            \begin{tabular}{|c||p{30em}|}\hline
                関数名&void TerminateServer(void)  \\\hline
                機能&サーバーの終了処理を行う\\
                引数&無し\\
                返り値&無し\\\hline
            \end{tabular}
        \end{center}
    \end{table}
    \item システムモジュール
    \begin{table}[H]
        \label{table:fanc_s2-1}
        \begin{center}
            \begin{tabular}{|c||p{30em}|}\hline
                関数名&const PlayerData* GetPlayerData()\\\hline
                機能&クライアントデータ配列の先頭ポインタを返す\\
                引数&無し\\
                返り値&無し\\\hline

            \end{tabular}
        \end{center}
    \end{table}
    \begin{table}[H]
        \label{table:fanc_s2-2}
        \begin{center}
            \begin{tabular}{|c||p{30em}|}\hline
                関数名&void GetClientName(int id, char clientName\verb+[MAX_LEN_NAME]+)\\\hline
                機能&クライアントの名前の取得\\
                引数&id ―― 送信先のクライアントID\\
                &clientName\verb+[MAX_LEN_NAME]+ ―― クライアントの名前\\
                返り値&無し\\\hline

            \end{tabular}
        \end{center}
    \end{table}
    \begin{table}[H]
        \label{table:fanc_s2-3}
        \begin{center}
            \begin{tabular}{|c||p{30em}|}\hline
                関数名&void GetPosition(int charaID, FloatPosition pos)\\\hline
                機能&クライアントの座標の取得\\
                引数&charaID ―― 受け取るクライアントのID\\
                &pos ―― クライアントの座標\\
                返り値&無し\\\hline
            \end{tabular}
        \end{center}
    \end{table}
\end{enumerate}

\subsubsection{クライアント}



以下に各モジュールの外部関数を説明する
\begin{enumerate}
    \item ネットワークモジュール
    \begin{table}[H]
        \label{table:fanc_c1-1}
        \begin{center}
            \begin{tabular}{|c||p{30em}|}\hline
                関数名&void SetupClient(char *serverName, u\_short port) \\\hline
                機能&クライアントの初期設定を行う\\
                引数&serverName―― サーバー名\\
                &u\_short port ―― ポート番号\\
                返り値&無し\\\hline
            \end{tabular}
        \end{center}
    \end{table}
    \begin{table}[H]
        \label{table:fanc_c1-2}
        \begin{center}
            \begin{tabular}{|c||p{30em}|}\hline
                関数名&int ControlRequests(void) \\\hline
                機能&クライアントの制御を行う\\
                引数&無し\\
                返り値&1:通信継続/0:通信終了\\\hline
            \end{tabular}
        \end{center}
    \end{table}
    \begin{table}[H]
        \label{table:fanc_c1-3}
        \begin{center}
            \begin{tabular}{|c||p{30em}|}\hline
                関数名&int InCommand(char com) \\\hline
                機能&コマンドに対する処理を行う\\
                引数&com ―― コマンド\\
                返り値&1:通信継続/0:通信終了\\\hline
            \end{tabular}
        \end{center}
    \end{table}
    \begin{table}[H]
        \label{table:fanc_c1-5}
        \begin{center}
            \begin{tabular}{|c||p{30em}|}\hline
                関数名&void TerminateClient(void)  \\\hline
                機能&クライアントの終了処理を行う\\
                引数&無し\\
                返り値&無し\\\hline
            \end{tabular}
        \end{center}
    \end{table}
    

    \item 入力モジュール
    \begin{table}[H]
        \label{table:fanc_c2-1}
        \begin{center}
            \begin{tabular}{|c||p{30em}|}\hline
                関数名&virtual void GetInput(SDL\_Event)\\\hline
                機能&ユーザー入力を受け取る\\
                引数&event ―― SDLのイベント\\
                返り値&無し\\\hline
            \end{tabular}
        \end{center}
    \end{table}
    \begin{table}[H]
        \label{table:fanc_c2-2}
        \begin{center}
            \begin{tabular}{|c||p{30em}|}\hline
                関数名&InputType GetInputType()\\\hline
                機能&現在の入力情報を返す\\
                引数&無し\\
                返り値&\_Input: 現在の入力情報\\\hline
            \end{tabular}
        \end{center}
    \end{table}
    
\end{enumerate}



\end{document}
